\documentclass[titlepage]{article}
\author{Thomas van Maaren (4216474)}
\title{Portfolio}

\begin{document}
\maketitle{}
\tableofcontents
\section{E-mails}
\subsection{First e-mail}

\textbf{Subject:} Applying for “Business Analyst Logistics”

\ 
\\
Dear Sir/Madam,

\ 
\\
Your vacancy for a “Business Analyst Logistics” is very interesting as I
have a passion for data and the shipping industry and have the skills and
experience required for this post.

\ 
\\
My name is Thomas van Maaren and I am a Applied Mathematics student. I am
also employed as as a maths tutor and I program in my free time.  In the attachments you can
find my CV and my cover letter.

\ 
\\
I am very much looking forward to your reply,

\ 
\\
Yours faithfully,
\\
Thomas van Maaren
\subsection{Response invitation interview}

Dear Ms. Jensen,

\ 
\\
Thank you very much for your reply and the invitation for the interview. I can
confirm that I am available on June the 24th at 11:45 a.m.

\ 
\\
I am grateful for the opportunity to talk with you and am looking forward to
finding out more about the position.

\ 
\\
Yours sincerely,
\\
Thomas van Maaren
\section{Curriculum Vitae}
\subsection{PERSONAL DETAILS}

\begin{tabular}{l l}

	Name:			&	Thomas van Maaren \\
	Town of residence:	&	Culemborg, The Netherlands \\
	E-mail:			&	thomas.v.maaren@outlook.com \\
	Date of Birth:		&	19 December, 2002 \\
	Nationality:		&       Dutch, British \\
	Gender:			&	Male \\
	Marital status:		&	Single \\
	Driving license:	&	None \\

\end{tabular}

\subsection{EDUCATION AND QUALIFICATIONS}

\begin{tabular}{l p{10cm}}
	2020-		&Applied Mathematics, Fontys University of Applied Science (Eindhoven).\\
	2017-2020	&Secondary School, Axia College (Amersfoort).\\

\end{tabular}

\subsection{WORK EXPERIENCE}

\begin{tabular}{l p{10cm}}
	2020-		&Lyceo. Tutoring Mathematics to students in secondary school\\
	2020-		&Education committee at Fontys. Coordinate
			questionnaires. Send the questionnaires to the students and summarise
			feedback for the staff.\\
\end{tabular}


\subsection{SKILLS}

%Seperate by competence
\textbf{Computer programmes:} Vim, Microsoft Excel, Aimms, Linux, Git.
\\
\textbf{Computer languages:} A lot of experience in C, Python, Latex; some
experience in C++, HTML, Matlab, GNU Octave.
\\
\textbf{Spoken Languages:} Dutch (first language), English (near native)

\subsection{INTERESTS}

Drama (member of a drama group); Piano (follow piano lessons every
week), Cycling, Camping, Computer Science, Mathematics.

\subsection{REFERENCES}

References are available upon request.

\section{Cover letter}
Dear Sir/Madam,

\ 
\\
This letter is about the vacancy for “Business Analyst Logistics”
on your website. I would appreciate it if I could be considered for this post.

\
\\
For the past year I have been a Applied Mathematics student. I also like to
write software in my free time because of this I have gained knowledge about a
varied array of subjects and am able to work on my own initiative.

\ 
\\
Currently, I am employed with Lyceo as a maths tutor for secondary school
students. I am also in the education committee of my school, Fontys. In this
post I have been given the responsibility of coordinating questionnaires. This
includes sending the questionnaires to the students and summarising feedback
for the staff.  

\ 
\\
The reason I am particularly interested in working for this company is because I
hold a keen interest in the shipping industry. I would therefore be very
interested in working in this industry.

\ 
\\
I hope this letter will give me the opportunity to be invited to an interview.
I look forward to your reply and would be pleased to provide any further
information.

\ 
\\
Yours faithfully,
\\
Thomas van Maaren
\section{Final report}
\subsection{Introduction}

Being someone with quite an international background I have had to travel quite
a lot in my youth. Every year we would take the ferry to Newcastle from
Ijmuiden to see our family. I always enjoyed this journey because it gave me
an interest in the shipping industry and in transport and logistics in general.
This is why I am excited to work for this business specifically. It will make
it possible for me to combine my interest in shipping with my interest and
skill in mathematics.

Because I had already sailed with DFDS Seaways I was already aware of this
company and decided to apply for business intelligence because this is the
field which I specialised in during my studies.

I started by trying to find a vacancy that would be interesting for me. First
of all I looked on the website of DFDS seaways because I had already sailed with this
shipping company in my youth. On the website I found a
few vacancies that would fit with my study. A while later I decided to go for
``Business analyst logistics" as it was the job in which I could utilise my
analytical skills and IT skills the most. A few days after having chosen
this position I thought this might not be the right one. This was a position
that would fit very well with Applied Mathematics, but it might not be my
``dream job". I found a vacancy at KHRONOS, which is a
company that works with computer graphics. However, when I asked my teacher if it
would be alright to choose a vacancy that has requirements that deviate from the
scope of the study, he said it was not. Then I decided to go back to DFDS.  When
I had finally decided on the vacancy I worked on my CV. I chose the British
format because I have a bigger connection with that culture.  Then I wrote a
cover letter with a request to be considered for the post of ``Business analyst
logistics". I wrote a simple e-mail with the cover letter and CV attached and
sent this to DFDS.

In this job I hope to develop my communication skills. I will be working in an
international environment and will need to function as a bridge between the
software department and the development department. I also expect to gain
experience in working with the newest IT tools 
\subsection{Body}

DFDS is a Danish shipping company founded in 1866. DFDS is structured into two
divisions. The first division is its ferry division which is responsible for
moving passengers and freight around Europe.  Then there is the logistics
division which is responsible for all land transport, warehousing and
logistical services in general. DFDS is working to digitise their services.
This will make it easier for customers to manage their bookings and make it
easier for DFDS to optimise their operations as more data will be available.

Nowadays DFDS is mostly focused on transportation throughout Europe. It offers
37 routes, nine of them being for passengers. They also provide logistic solutions.
This includes the storage of goods and land travel.

For this position I will be working at the DFDS headquarters in Copenhagen. I will
function as the link between the developers and the operational staff. This I
will do in SCRUM, which is a mindset about working on projects. It is my
responsibility to design an optimal solution for a given feature of the
business. This means that I will analyse processes that are running in the
business and try to find a better way to perform them. Then I will present
this improved method, giving arguments on why this will create a better value for
the company. I will also help with the implementation of the new method and
give a follow up on the performance. For this it is necessary to understand
the logistical processes that happen inside of DFDS.

Applied mathematics is very useful for this position. In particular, the
subjects Operations Research and Data Analysis. Operations Research 
will teach me about logistical processes in companies and optimisation
techniques and Data Analysis will give me the capability to interpret
big chunks of data, which is necessary when measuring the value of a certain
process.

Soft skills are also important. This company is very international, meaning
that I will have to speak in English and interact with cultures that I am not
accustomed to. I will be doing a lot of presentations and will have a lot of
discussions with my collogues. It is therefore important that I have good
international communication skills, good presentation skills and can work well in a team.

\subsection{Conclusion}
%I started by trying to find a vacancy that would be interesting for me. First
%of all I looked on the website of DFDS seaways. This is a shipping company
%which I had sailed with a lot in my youth so I already knew of the company.
%There I found a few vacancies that would fit with my study. A while later I
%decided to go for ``Business analyst logistics" as it was the job where I could
%utilise my analytical skills and IT skills the most. A few days after I had
%already chosen this position I thought this might not be the right one. This
%was a position that would fit very well fit Applied Mathematics, but it
%might not be my ``dream job" as you might call it. I found a vacancy at
%KHRONOS. Which is a company that works with computer graphics. But when I asked
%my teacher if it would alright to choose a vacancy that has requirements that
%deviate from the scope of the study he said it wasn't. Then I decided to go
%back to DFDS. 
%
%When I had finally decided on the vacancy I worked on my CV. I Chose the
%British format, because I have a bigger connection to that culture.
I found this exercise educational not only because I learned how I should apply
for a job, but also because it gave me a better insight in the jobs available to me
when I finish this study. For project 4 we had to use SCRUM and in this
exercise I discovered that this is actually used in the workplace. This has
made me more enthusiastic about my future career.


\end{document}

% vim: cc=100
